\documentclass[16pt,a4paper]{article}
\usepackage[utf8]{inputenc}
\usepackage{polski}
\usepackage[polish]{babel}
\usepackage{amsmath}
\usepackage{amsfonts}
\usepackage{amssymb}

%foot
\usepackage{fancyhdr}
\pagestyle{fancyplain}
\fancyhf{}
\renewcommand{\headrulewidth}{0pt}
\renewcommand{\footrulewidth}{0.4pt}
\fancyfoot[L]{Patryk Lisik, Adam Sadowski, Projekt wstępny -- \textit{Cukiernia pod wanilinowym uśmieszkiem}}
%\fancyfoot[R]{\thepage\ / \pageref{LastPage}}


\begin{document}
\begin{titlepage}

\begin{minipage}{0.33 \textwidth}
\begin{flushleft}
\large
Nr. albumu: \textsc{210254}\linebreak
Imię i nazwisko:\\
\textsc{Patryk Lisik}
\end{flushleft}
\end{minipage}
\hspace{0.2\textwidth}
\begin{minipage}{0.33 \textwidth}
\begin{flushleft}
\large
Nr. albumu: \textsc{210310}\linebreak
Imię i nazwisko:\\
\textsc{Adam Sadowski}
\end{flushleft}
\end{minipage}

\vspace{3cm}
\begin{minipage}{0.5\textwidth}
\begin{flushleft}
\Large
Kierunek: \textsc{Informatyka} \\
rok akademicki: \textsc{2018/2019} \\
grupa \textsc{wtorek 12:30} \linebreak\linebreak
\end{flushleft}
\end{minipage}

\vspace{3cm}

{\center\huge\bfseries Projektowanie systemów informatycznych \par}
\vspace{1.5cm}
{\center\huge\bfseries Projekt wstępny \par}
{\center\Large\itshape Cukiernia pod wanilinowym uśmieszkiem \par}

\end{titlepage}
\section{Wprowadzenie}
\subsection{Cel dokumentu}
Celem dokumentu jest zdefiniowanie i opisanie zadań stawianych przed system informatycznym sieci cukierni. Dokument jest jedynie mglistym przybliżeniem opisywanego systemu i nie ma za zadanie podawania dokładnych sposobów realizacji przedstawionych wymagań.
\subsection{Zakres produktu}
System ma zapewnić kompleksową obsługę sieci małych cukierni. System ma obsługiwać wszystkie wejścia jak dostawy towaru, płatności czy pensje. Ponadto ma mieć możliwość generowania raportów z podanego okresu czasu określających ile i jakiego typu produkty były najpopularniejsze w danym czasie oraz przedstawić bieżący stan zaopatrzenia w magazynach i sklepach. 
\subsection{Definicje akronimy i skróty}
\subsection{Odwołania do literatury}
\subsection{Omówienie dokumentu}
\section{Opis ogólny}
\subsection{Kontekst funkcjonowania}
\subsection{Charakterystyka użytkowników}
\subsection{Główne funkcje produktu}
\subsection{Ograniczenia}
\subsection{Założenia i zależności}
\section{Wymagania funkcjonalne}
\section{Wymagania niefunkcjonalne}
\end{document}