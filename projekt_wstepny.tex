\documentclass{article}
\usepackage[utf8]{inputenc}
\usepackage{amssymb}
\usepackage{polski}
\usepackage[polish]{babel}
\usepackage{amsmath}
\usepackage{amsfonts}
\usepackage{hyperref}
\usepackage{lastpage}
\usepackage{graphicx}
\usepackage{bookmark}

%foot
\usepackage{fancyhdr}
\pagestyle{fancyplain}
\fancyhf{}
\renewcommand{\headrulewidth}{0pt}
\renewcommand{\footrulewidth}{0.4pt}
\fancyfoot[L]{Patryk Lisik, Adam Sadowski \textit{Cukiernia pod wanilinowym uśmieszkiem}}
\fancyfoot[R]{\thepage\ / \pageref{LastPage}}


\begin{document}

\begin{titlepage}

\begin{minipage}{0.33 \textwidth}
\begin{flushleft}
\large
Nr. albumu: \textsc{210254}\linebreak
Imię i nazwisko:\\
\textsc{Patryk Lisik}
\end{flushleft}
\end{minipage}
\hspace{0.2\textwidth}
\begin{minipage}{0.33 \textwidth}
\begin{flushleft}
\large
Nr. albumu: \textsc{210310}\linebreak
Imię i nazwisko:\\
\textsc{Adam Sadowski}
\end{flushleft}
\end{minipage}

\vspace{3cm}

\begin{minipage}{0.9\textwidth}
\begin{flushleft}
\Large
Kierunek: \textsc{Informatyka -- SIBD} \\
rok akademicki: \textsc{2018/2019} \\
grupa \textsc{wtorek 12:30} \linebreak\linebreak
\end{flushleft}
\end{minipage}

\vspace{3cm}

{\center\huge\bfseries Projektowanie systemów informatycznych \par}
\vspace{1.5cm}
{\center\huge\bfseries Projekt wstępny \par}
{\center\Large\itshape Cukiernia pod wanilinowym uśmieszkiem \par}

\end{titlepage}
\section{Wprowadzenie}
\subsection{Cel dokumentu}
Celem dokumentu jest zdefiniowanie i opisanie zadań stawianych przed system informatycznym sieci cukierni. Dokument jest jedynie mglistym przybliżeniem opisywanego systemu i nie ma za zadanie podawania dokładnych sposobów realizacji przedstawionych wymagań.
\subsection{Zakres produktu}
System kierowany jest do:
\begin{itemize}
    \item Klientów -- umożliwiając im wygodne zakupy wyrobów
    \item Pracowników cukierni -- przyśpieszając i ułatwiając ich pracę
    \item Managerów cukierni --  umożliwiając łatwe monitorowanie stanu sklepów i magazynów 
\end{itemize}
\subsection{Definicje akronimy i skróty}
\begin{enumerate}
    \item Visa
    \item MasterCard
    \item AmericanExpress
    \item BLIK
    \item GPay
    \item ApplePay
    \item Magazynier
    \item Kasjer
    \item Klient
    \item Promocja
    \item Produkt niezdatny do sprzedaży
\end{enumerate}
\subsection{Odwołania do literatury}
\subsection{Omówienie dokumentu}
\section{Opis ogólny}
\subsection{Kontekst funkcjonowania}
System ma zapewnić kompleksową obsługę sieci małych cukierni.
System ma obsługiwać wszystkie wejścia jak dostawy towaru, płatności czy pensje. 
Ponadto ma mieć możliwość generowania raportów z podanego okresu czasu określających ile i jakiego typu produkty były najpopularniejsze w danym czasie oraz przedstawić bieżący stan zaopatrzenia w magazynach i sklepach. 

\subsection{Charakterystyka użytkowników}
Przewidywani użytkownicy sytemu to:
\begin{enumerate}
\item Klient
\item Kasjer
\item Magazynier
\item Manager 
\end{enumerate}
\subsection{Główne funkcje produktu}
\begin{itemize}
    \item Sprzedaż
    \begin{itemize}
        \item  Bezpośrednio w cukierni
        \item  Poprzez stronę internetową
    \end{itemize}

    \item Przyjmowanie dostaw
    \item Automatyczne wykonywanie przelewów z pensjami
    \item Obsługa promocji
    \item Informowanie o aktualnym stanie magazynów i sklepów
\end{itemize}
\subsection{Ograniczenia}
System musi spełniać założenia zgodne z rozporządzeniem Parlamentu Europejskiego i Rady (UE) 2016/679 z 27.04.2016 r. w sprawie ochrony osób fizycznych w związku z przetwarzaniem danych osobowych i w sprawie swobodnego przepływu takich danych oraz uchylenia dyrektywy 95/46/WE (ogólne rozporządzenie o ochronie danych) (Dz. U. UE. L. z 2016 r. Nr 119, str. 1).

\subsection{Założenia i zależności}

\section{Wymagania funkcjonalne}
\begin{enumerate}
\item Sprawdzanie aktualnego stanu zaopatrzenia magazynu i poszczególnych cukierni.
\item Generowanie raportów z wybranego okresu czasu.
\item Konfigurowanie zawartości raportów (wybór sklepów i magazynów, wybór typów produktów).
\item Obsługa magazynów(dostawy i towaru płatności).
\item Obsługa kas(sprzedaż przyjmowanie płatności).
\item Obsługa kart płatniczych Visa, MasterCard, AmericanExpress oraz systemów płacenia telefonem BLIK, GPay, ApplePay
\item Automatyczne wykonywanie przelewu z wynagrodzeniem dla pracowników.
\item Zamawianie przez wyrobów przez stronę internetową.
\item Dodawanie promocji na dany okres czasu.
\item Usuwanie ze stanu zaopatrzenia produktów niezdatnych do sprzedaży.
\end{enumerate}
\section{Wymagania niefunkcjonalne}
\begin{enumerate}
\item Każdy produkt musi mieć cenę wyższą niż 0.01 zł
\item Promocja wyrażona w procentach musi zwierać się w przedziale 5\%-95\% włącznie.
\item System musi działać przez 99.99\% roku(może nie działać przed 518min) z wyłączaniem przerw na konserwację.
\item Czas uruchomiania poniżej 5 min.
\item Czas od kliknięcia do otrzymania potwierdzania o przyjęciu zmówienia nie większy niż 45 sekund. 
\end{enumerate}

\end{document}
